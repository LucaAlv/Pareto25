\subsection{Migration Data}
The migration data used in this paper was sourced from the Mexican Population and Housing Census for the years 2000, 2010, 2015 and 2020. Specifically of interest was the long-form survey that was administered to approximately 10\% of households and included an extended questionnaire with information relevant for inter-municipal migration flows. The sampling process, i.e. the process of selecting which household received the extended questionnaire is based on multistage probability sampling with slight differences in the exact methodology between the three rounds. The actual data used was downloaded from the IPUMS International Database \cite{Minnesota Population Center}. This greatly facilitates replication and ensures standardized variable naming across censuses. \\

During pre-processing, rows with invalid municipality codes, missing information on their municipality of residence five years ago, and missing information on the demographic data were removed. This lowered the amount of observations to around 42 million individual-level observations for the three census-iterations 2010, 2015, and 2020. 

Table 1 reports summary statistics aggregated across census rounds. A similar table with summary statistics for each census-round can be found in table 1A in the appendix. In Table 1 summary statistics on the migration outcomes are reported in panel A. Note that a "N" of 6975 indicates that data is available for all 2325 (IPUMS harmonized) municipalities for the three census rounds. Across time, on average in- and out-migration balances out, leading to an average net migration of 0. Rates were calculated by dividing absolute migration numbers by population size at the time where the census data was collected. 

\subsection{Demographic Data}
The census data that was additionally used to extract key demographic variables that serve as controls in the regressions below. During pre-processing, invalid municipality codes were removed and municipality level, weighted means of relevant variables, such as age, literacy-rate, male/female ratio, etc. were calculated. The respective weights (contained in the perwt variable in the data) were chosen to make sure that averages represent the total Mexican population accurately. Summary statistics for the demographic data are provided in Panel C of Table 1. 

\subsection{Climate Data}

The National Center for Prevention of Disasters (CENAPRED) publishes data on the number of emergencies, disasters and climatic contingencies for every year and municipality.\cite{CENAPRED2024} This information is collected trough analysis of the government's official newspaper, which among other things publishes any declarations of emergency for one or more municipalities by the federal government. Such a declaration of emergency follows the presence of a severe disturbing natural agent in certain municipalities or delegations of one or more federal entities, whose damage exceeds the local financial and operational capacity. Declaring this emergency is necessary in order to be able to access resources of the financial instrument for the care of natural disasters, such as the Natural Disasters Fund. 
only a registry of official disaster declarations
The Atlas "Declaratorias" offers a very powerful and convenient count of disastrous weather events on a municipality level. Since it only records weather events of whom local authorities thought to be disastrous enough to apply for federal help, the data indirectly also filters for weather events that had a significant impact on local municipalities. On the other hand this pre-selection of weather events also makes clean inference more difficult. In addition to the information on the disaster relevant for this analysis, they also reflect administrative capacity to request and obtain a declaration from the federal government, hidden political incentives, eligibility rules and the amount of coordination between local and federal governments. So coefficients may be picking up a combiantion of disaster damage, local capacity and attractiveness of reporting. 

This data is available from 2000 to 2024 on a municipality level. During pre-processing the current municipality codes used by the CENAPRED data are matched to the harmonized version used by IPUMS. As IPUMS combines some municipalities to maintain a harmonized structure across multiple census rounds, a small amount of information is lost here. The disaster data are aggregated to five year intervals, to match the census time intervals. This means that the effect of all events recorded e.g. during 2001 and 2005 are attributed to the 2005 census. As the census in 2005 captures all movement between 2000 and 2005, the idea behind this is that events recorded between 2000 and and 2005 also influence migration in this time period. This of course is an oversimplification - a tropical storm that hit land on the 31st December of 2005 may very well only start having an effect on migration in 2006 or later. Furthermore, as the dataset conveniently also records the type of event a counter for each type of disaster is extracted from the data.

\subsection{Crime Data}
Next to extreme weather events, the main driver of displacement and short-term migration in Mexico is violent crime. A key indicator for crime is inferred from the amount of homicides in a municipality. As with the disaster data events are aggregated to the five year-census intervals. \cite{crime data}


\begin{table}[!h]
\centering
\caption{Summary statistics (municipality-year level)}
\centering
\fontsize{9}{11}\selectfont
\begin{threeparttable}
\begin{tabular}[t]{lrrrrr}
\toprule
Variable & N & Mean (SD) & Median & Min & Max\\
\midrule
\addlinespace[0.3em]
\multicolumn{6}{l}{\textbf{Panel A: Migration outcomes}}\\
\hspace{1em}Period Immigration & 6975 & 2512.33 (11059.83) & 344.00 & 0.00 & 276699.00\\
\hspace{1em}Period Emmigration & 6975 & 2512.33 (13784.52) & 386.00 & 0.00 & 483051.00\\
\hspace{1em}Net Immigration & 6975 & 0.00 (8410.35) & -4.00 & -219655.00 & 113326.00\\
\hspace{1em}Immigration Rate & 6975 & 0.04 (0.03) & 0.03 & 0.00 & 0.67\\
\hspace{1em}Emmigration Rate & 6975 & 0.04 (0.05) & 0.03 & 0.00 & 2.10\\
\hspace{1em}Net Immigration Rate & 6975 & -0.00 (0.06) & -0.00 & -2.06 & 0.66\\
\addlinespace[0.3em]
\multicolumn{6}{l}{\textbf{Panel B: Disaster exposure}}\\
\hspace{1em}Total Unique Disasters (Period) & 6975 & 3.79 (4.79) & 2.00 & 0.00 & 86.00\\
\hspace{1em}Total Disasters (Period) & 6975 & 3.26 (4.02) & 2.00 & 0.00 & 78.00\\
\addlinespace[0.3em]
\multicolumn{6}{l}{\textbf{Panel C: Controls}}\\
\hspace{1em}Mean Income & 6965 & 3313.14 (2378.09) & 2941.76 & 34.95 & 39125.43\\
\hspace{1em}Mean Age & 6975 & 30.99 (3.71) & 30.66 & 20.24 & 48.91\\
\hspace{1em}Female Rate & 6975 & 0.51 (0.02) & 0.51 & 0.43 & 0.61\\
\hspace{1em}Literacy Rate & 6966 & 0.86 (0.08) & 0.88 & 0.42 & 1.00\\
\hspace{1em}Mean Years of Schooling & 6966 & 6.08 (1.28) & 6.01 & 2.07 & 12.62\\
\hspace{1em}Municipal Population & 6975 & 51188.19 (200573.91) & 12704.00 & 76.00 & 5413803.00\\
\hspace{1em}Population Density & 6975 & 219.80 (719.69) & 51.75 & 0.11 & 10714.21\\
\bottomrule
\end{tabular}
\begin{tablenotes}
\item \textit{Notes: } 
\item Unit of observation is municipality-year. N reports non-missing observations.
\end{tablenotes}
\end{threeparttable}
\end{table}

\subsection{Data merging}
After pre-processing the data (i.e. removing NAs and matching geographic data, where necessary) the datasets for migration, disasters, and demographics were merged into two datasets:  
Firstly, a municipality-node dataset, that contains one entry for each node (=municipality) and census year combination. This dataset is relevant for the TWFE estimation. Secondly, a municipality-edge dataset that contains information about the amount of people who migrated from one municipality to another in a certain time span. This dataset is relevant for the gravity model.