The following gives a short overview of the sampling process for the 2010 Census:
Municipalities were stratified into three groups by number of inhabited dwellings:
    Fewer than 1,100
    1,100 to 4,000
    More than 4,000
All dwellings in municipalities with fewer than 1,100 dwellings were fully included in the sample. All dwellings in the 125 municipalities with the lowest human development index were fully sampled with probability 1. For remaining areas, primary sampling units (PSUs) were selected. In localities with fewer than 250 dwellings, the entire locality served as the PSU. In localities with fewer than 50,000 inhabitants, these were divided into nine strata by population size; blocks served as PSUs. In localities with 50,000 or more inhabitants, PSUs were “basic geostatistical areas,” consisting of clusters of contiguous city blocks with similar population sizes. The Mexican Statistical Agency INEGI established minimum sample sizes for each stratum, ranging from 800 dwellings for municipalities with 1,100–4,000 dwellings to 2,000 dwellings for localities with over 50,000 inhabitants.
Sample sizes were adjusted based on dwelling counts and finite population calculations to enable accurate estimation of municipal characteristics with proportions $\geq$0.01, including migration flows (average = 6\%).