\subsection{Motivation and Research Question}
Mexico is among the countries most exposed to extreme climate events in the Americas. Its geographic location between the Pacific and Atlantic basins places it directly in the path of tropical cyclones. In addition, large internal climatic variation makes many regions vulnerable to droughts, floods, wildfires, and heatwaves. Over the past two decades, the frequency and severity of these events has increased, amplifying their economic and social consequences. According to Mexico’s National Institute of Statistics and Geography (INEGI), economic losses from hydrometeorological disasters regularly amount to billions of pesos annually; for example, Hurricane Odile (2014) and Hurricane Wilma (2005) alone generated damages equivalent to roughly 1.5\% and 2.5\% of national GDP, respectively (\cite{CENAPRED, CENAPRED2016}). The National Center for Disaster Prevention (CENAPRED) further documents long-run upward trends in both the number of reported climate-related disasters and their associated financial losses (\cite{CENAPRED, CENAPRED2020}).\\

[Figure 1: Trend in climate-related disaster damages in Mexico, 2000–present] \\

Beyond direct economic losses, extreme climate events shape demographic patterns by disrupting livelihoods, damaging infrastructure, and altering local labor markets. The Internal Displacement Monitoring Centre (IDMC) reports that Mexico experiences tens of thousands of new disaster-related displacements each year, driven primarily by storms and floods (IDMC, 2023).\\

[Figure 2: Annual disaster-related displacements in Mexico, IDMC data] \\

While there is clear data on displacement incidents within shorter time periods, long term migration patterns are often more complex. Permanent migration decisions are influenced by a multitude of factors and people who are temporarily displaced often return to their original homes. Municipalities within Mexico experience heterogeneous exposure to climate hazards and differ significantly in adaptive capacity, economic structure, and institutional strength. Understanding the relationship between disasters and long-term migration is crucial, not only for improving disaster-risk management but also for informing climate adaptation and regional development policy. \\

This seminar paper examines how extreme climate events affect long term internal migration patterns. To estimate long-term migration a panel of individual-level movement data from 2005 to 2020 was created using data from the 2010, 2015 and 2020 Mexican Census. Leveraging data for the extended study period of 15 years increases the likelihood of estimating coefficients that are independent from the effects of cyclical trends and short-term displacement. In addition, data on the amount of disasters per time period is taken from the Center of Disaster and Risk Prevention (CENAPRED).

Similarly, to the approaches used by Thiede, Gray, and Mueller (2016) and Mueller, Gray, and Hopping (2020) this paper uses a fixed-effects regression model to estimate the effects of the amount of extreme-weather events on long-term migration outcomes in Mexico. This makes it possible to control for time-invariant municipality fixed effects, aswell as municipality-invariant period- (i.e. the 5-year census-intervals) fixed effects. 