\subsection{Motivation and Research Question}
Mexico is among the countries most exposed to extreme climate events in the Americas. Its geographic location between the Pacific and Atlantic basins places it directly in the path of tropical cyclones. In addition, large internal climatic variation makes many regions vulnerable to droughts, floods, wildfires, and heatwaves. Over the past two decades, the frequency and severity of these events has increased, amplifying their economic and social consequences. According to Mexico’s National Institute of Statistics and Geography (INEGI), economic losses from hydrometeorological disasters regularly amount to billions of pesos annually; for example, Hurricane Odile (2014) and Hurricane Wilma (2005) alone generated damages equivalent to roughly 1.5\% and 2.5\% of national GDP, respectively (\cite{CENAPRED, CENAPRED2016}). The National Center for Disaster Prevention (CENAPRED) further documents long-run upward trends in both the number of reported climate-related disasters and their associated financial losses (\cite{CENAPRED, CENAPRED2020}).\\

[Figure 1: Trend in climate-related disaster damages in Mexico, 2000–present] \\

Beyond direct economic losses, extreme climate events shape demographic patterns by disrupting livelihoods, damaging infrastructure, and altering local labor markets. Mexico has experienced recurrent climate-induced displacement, particularly in rural and coastal areas where households depend heavily on climate-sensitive income sources such as agriculture, fisheries, and tourism. Empirical studies find that drought conditions and temperature anomalies significantly affect internal migration flows: for instance, \cite{Nawrotzki2017} show that rainfall deficits increase the likelihood of rural out-migration, while \cite{Feng2012, Oppenheimer, & Schlenker (2012)} identify similar climate–migration channels across Mexico’s agricultural sector. Hurricanes and floods have also been linked to short-term displacement and longer-term resettlement decisions, especially in southern states such as Chiapas, Veracruz, and Tabasco (\cite{Vásquez-León et al., VasquezLeon2018}; \cite{CONAPO, CONAPO2020}). \\

[Table 1: Summary of major extreme events and estimated displacement figures] \\

Internal displacement associated with climate shocks is not only common but increasingly measurable. The Internal Displacement Monitoring Centre (IDMC) reports that Mexico experiences tens of thousands of new disaster-related displacements each year, driven primarily by storms and floods (IDMC, 2023). These patterns suggest that extreme climate events function as both acute and chronic stressors: acute events force short-term evacuations, while chronic climatic pressures—such as prolonged drought or repeated flooding—contribute to gradual mobility decisions and long-term migration reallocation. \\

[Figure 2: Annual disaster-related displacements in Mexico, IDMC data] \\

Despite this growing evidence, important gaps remain. Municipalities within Mexico experience heterogeneous exposure to climate hazards and differ markedly in adaptive capacity, economic structure, and institutional strength. Yet we still lack systematic, municipality-level evidence on how specific types of extreme events—storms, floods, wildfires, and droughts—shape internal migration flows over time. Understanding this relationship is crucial, not only for improving disaster-risk management but also for informing climate adaptation, regional development policy, and social protection strategies. \\

This seminar paper contributes to this literature by examining how extreme climate events affect internal migration patterns across Mexican municipalities from 2000 onward. Leveraging high-resolution climate hazard datasets and panel data on inter-municipal migration flows, I aim to quantify the extent to which climate shocks trigger local population mobility.